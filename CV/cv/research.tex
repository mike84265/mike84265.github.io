%-------------------------------------------------------------------------------
%	SECTION TITLE
%-------------------------------------------------------------------------------
\cvsection{Research Projects}


%-------------------------------------------------------------------------------
%	CONTENT
%-------------------------------------------------------------------------------
\begin{cventries}
    \cvproject
    {A Performance Analytical Model for DNN Training with Focus on Memory Subsystem}
    {MS Thesis}
    {Exploring the tradeoffs between cache capacity and memory bandwidth in DNN workloads}
    {Feb '19 - Jun '21}
    {
        \begin{cvitems}
        % \item The first work to model data reuse across DNN layers in DNN workloads for 100+ MB cache % (AMD has products with this scale already).
        \item Devised a software-controlled cache scheme model data reuse across DNN layers in DNN workloads.
        % \item Devised a novel software-controlled cache analytical scheme to approximate an optimal hardware design so that the architecture problem can be decoupled from low-level design issues.
        \item Analyzed tradeoffs between cache capacity and bandwidth in ResNet, MobileNet, GNMT, and Transformer and drew some insights from the experiment results.
        \end{cvitems}
    }

    \cvproject
    {Standard Cell Delay Calculator}
    {Internal Project at TSMC}
    {A model to swiftly estimate the delay of a post-layout standard cell (MOS + parasitic R/C)}
    {Jun '22 - Jul '23}
    {
        \begin{cvitems}
        \item More than 10x faster than SPICE transient simulation, 100x faster than bi-section setup time simulation, while maintaining more than 0.9 of correlation coefficient and sub-ps accuracy with SPICE.
        \item Integrated graph theory, transient pre-characterization, layout dependent effect (LDE) estimation, asymptotic waveform evaluation, calculation on RC network and simplification, etc., into a single program to estimate propagation delay from a detailed standard parasitic format (DSPF) file.
        % \item Leveraged parallel programming, scientific computing, inter-process communication and binding between C++ and Python, and various techniques to assure optimal efficiency and ease of use.
        % \item Independently inquiring into influential factors while striking a balance between the model's efficiency and accuracy.
        \end{cvitems}
    }
\end{cventries}
